\documentclass[compress,blue]{beamer}

%\usepackage{beamerthemesplitcondensed}
\usepackage{beamertemplates}
\usepackage{beamerthemesplit}
\usepackage{beamerthemeshadow}
\usepackage[german]{babel}
\usepackage{pgf,pgfarrows,pgfnodes,pgfautomata,pgfheaps,pgfshade}
\usepackage{amsmath,amssymb}
\usepackage[latin1]{inputenc}
\usepackage{colortbl}
\usepackage{listings}
\usepackage{alltt}
\usepackage[T1]{fontenc}
\usepackage{ae,aecompl}

%\DeclareFontFamily{T1}{cmvtt}{}

\pgfdeclareimage[height=1cm]{unilogo}{logomitte}

\hypersetup{%
  pdftitle={Tracing von (Open-Source) Projekten mit Maven},%
  pdfauthor={Tammo van Lessen, Steffen Pingel},
  pdfsubject={inf.misc},
  pdfkeywords={Maven, Project Tracing, statcvs, xnap}}

\title[How to Mavenize your project]{Tracing von (Open-Source) Projekten mit Maven}
\author{Tammo van Lessen, Steffen Pingel}
\institute[Univerit�t Stuttgart]{
  Fakult�t f�r Elektrotechnik und Informatik\\
  Universit�t Stuttgart}
\date[inf.misc 2004]{inf.misc, 04.02.2004}

\logo{\pgfuseimage{unilogo}}

\beamertemplateshadingbackground{red!10}{structure!10}
%\beamertemplatetransparentcovereddynamic
\beamertemplatetransparentcovered
%\beamertemplateballitem
\beamertemplatesolidbuttons
\beamertemplateroundedblocks
\beamertemplatelargetitlepage

\begin{document}

\plainframe{\titlepage}

\section[Outline]{}
\frame{
  \begin{itemize}
  \item Basics
  \item Plugins
  \item Advanced Stuff
  \item Fazit
  \end{itemize}
}

\part{Maven Basics}
\plainframe{\partpage}
\section{Einfuehrung}
\subsection{Hintergrund}
%Maven was initially developed for buiding Turbine, Maven matured into an open source software engineering platform, The core functionality is automated project building, distribution and website creation 
%an easy way to publish project information and a way to share JARs across several projects.

\section{Installation}
\frame{
maven genapp
Verzeichnisstruktur
}
\section{Konzepte}
% 3 Kerndateien
% TARGET=GOAL (post/preGoal)
%A project is described with
%a XML Project Object
%Model (POM)
% The POM defines how to
%build a project and the
%external dependencies
% The Maven functionality is
%implemented in terms of
%plugins
% The plugins are written in
%Jelly
% JARs are downloaded
%from a remote repository
%and stored into a local
%repository
\subsection{Architektur}
%grafik
\subsection[POM]{Project Object Model}
% projekt eigenschaften
\subsection[PP]{Project Properties}
% customizing von plugins
\subsection{Repositories}

\section{Beispiel}

%%% Local Variables: 
%%% TeX-master: "index"
%%% End: 


\part{Maven Plugins}
\plainframe{\partpage}
\section{Core Plugins}

\subsection{�bersicht}
\frame{
  \frametitle{�bersicht}

  Die Core Plugins werden mit Maven ausgeliefert. Einige kennen wir bereits:
  
  \begin{itemize}
  \item clean
  \item jar
  \item java
  \end{itemize}

}

\subsection{ChangeLog}
\frame{
  \frametitle{ChangeLog Plugin}
}

\subsection{License}
\frame{
  \frametitle{License Plugin}
}

\section{Optional Plugins}

\subsection{�bersicht}
\frame{
  \frametitle{�bersicht}

  Die Optionalen Plugins werden bei Bedarf automatisch
  installiert. Auch hier haben wir einige bereits gesehen:

  \begin{itemize}
  \item genapp
  \item linkcheck
  \item release
  \item site
  \item xdoc
  \end{itemize}

  Auf der Maven Seite sind Ende Januar 2004 77 Plugins gelistet.
}

\subsection{Changes}
\frame{
  \frametitle{Changes Plugin}

  Versions Historie

  changes.xml
}

\subsection{Source Code Dokumentation}
\frame{
  \frametitle{Source Code Dokumentation}
  
  JavaDoc
  XRef
  JUnit
}

\subsection{Source Code Metriken}
\frame{
  \frametitle{Source Code Metriken}
  
   CheckStyle Plugin
   PMD Plugin
}

\subsection{Test�berdeckung}
\frame{
  \frametitle{Test�berdeckung}
  
   Clover
   JCoverage
   gcover
}

\subsection{CVS Statistiken}
\frame{
  \frametitle{CVS Statistiken}

  Developer Activity Plugin
  File Activity Plugin
}

\frame{
  \frametitle{StatCvs Plugin}
}

\subsection{Weitere}
\frame[all:1]{
  \frametitle{Noch Mehr N�tliche Plugins}

  \begin{itemize}
  \item ant
  \item console: Konsoleneingabe zum Aufruf von Goals
    \begin{itemize}
    \item (+) schnelles sukessives Aufrufen von Goals
    \item (-) unzureichende Speicherfreigabe f�hrt zu OutOfMemory
      exception
    \end{itemize}
  \item eclipse: Generierung von Eclipse Projekt Dateien
    \begin{itemize}
    \item Ber�cksichtigung aller Dependencies
    \item Anlegen der MAVEN\_HOME Variable in Eclipse notwendig
    \end{itemize}
  \item plugin
  \item release
  \item uberjar
  \end{itemize}
}


%%% Local Variables: 
%%% TeX-master: "index"
%%% End: 


\part{Unser Fazit}
\plainframe{\partpage}
\section{Zusammenfassung}

\frame{
  \frametitle{Zusammenfassung}

  \begin{columns}
    %\begin{column}{6cm}
	\begin{minipage}[t]{6cm}
  Maven ist ein {\bf brauchbares} Tool
  
  \begin{itemize}
  \item Sehr einfache Erstellung von Projektdokumentation
  \item Sehr einfache Erhebung von Projekt- und Prozess-Metriken
  \item Sehr einfaches Building und Releasing, Deploying 
  \item Automatische Aufl�sung von Abh�ngigkeiten
  \end{itemize}  
    \end{minipage}
    \begin{minipage}[t]{6cm}
  Manches ist noch {\bf nicht ausgereift}

  \begin{itemize}
  \item Lange Startzeiten (auf marvin ca. 4s)
  \item Viele Reports sind Java spezifisch
  \item Anpassung der Webseiten schwierig
  \item Keine I18n Unterst�tzung
  \item Generierte Webseiten enthalten viele Leerzeilen
  \item Aufruf aus ``falschen'' Verzeichnissen f�hrt zu {\it sehr} wirren Meldungen
  \end{itemize}
    \end{minipage}
  \end{columns}
}
  
\subsection{Resourcen}
\frame{
  \frametitle{Resourcen}

  \begin{itemize}
  \item http://maven.apache.org/
  \item http://maven.apache.org/reference/plugins/index.html
  \item http://jakarta.apache.org/commons/jelly/
  \item http://maven-plugins.sourceforge.net/
  \item http://wiki.codehaus.org/maven/
  \end{itemize}
}

%%% Local Variables: 
%%% TeX-master: "index"
%%% End: 


\plainframe{
\Huge{Vielen Dank!}
}

\end{document}
