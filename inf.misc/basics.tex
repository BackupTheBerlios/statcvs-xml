\section{Einfuehrung}
\subsection{Hintergrund}
%Maven was initially developed for buiding Turbine, Maven matured into an open source software engineering platform, The core functionality is automated project building, distribution and website creation 
%an easy way to publish project information and a way to share JARs across several projects.

\frame{
    \frametitle{Apache Maven}
    \begin{itemize}
        \item Build-Werzeug mit Projektverst�ndnis
        \item in Java entwickelt
        \item gestartet von Jason van Zyl als Subprojekt von Apache Turbine
        \item inziwschen Toplevel-Projekt bei der ASF
        \item Basiert auf Jelly (XML Scripting Engine) %vorher antbasiert
        \item Plugin-Architektur %java halt
    \end{itemize}
    
}

\frame{
    \frametitle{Ziele}
    \begin{itemize}
        \item einfach gestalteter Buildprozess
        \item Einheitliches Buildsystem 
        \item Informationserzeugung zum Projekt und dessen Qualit�t
        \item Providing guidelines for thorough testing practices 
Providing coherent visualization of project information 
Allowing transparent migration to new features 
   
    \end{itemize}
}

\frame{
    \frametitle{Konzepte}
    \begin{itemize}
        \item Goals (=Targets)
        \item Projekt-Metainformationen im sog. POM
        \item Anpassung s�mtlicher Parameter.
        \item Verwaltung aller Projekt-Dependencies in einem Repository
    \end{itemize}
}

\subsection{Verzeichnisstruktur}
\frame[all:1]{
    \frametitle{Verzeichnisstruktur}
    Maven w�nscht sich eine besondere Projektgliederung
    \scriptsize\begin{alltt}
statcvs-xml
  + src
    - conf
    + java
      - ...
    + test
      - ...
    - resources
    - (webapp)
  + target
    + docs
    - ...
  + xdocs
    - images
    - styles
  - maven.xml
  - project.xml
  - project.properties
    \end{alltt}
}

\subsection{Goals}
\frame[all:1]{
    \frametitle{Goals}
    \begin{itemize}
        \item Goals sind Funktionen im Buildprozess.
        \item in Jelly-XML definiert.
        \item Jedes Goal hat ein Pre- und Post-Goal
        \item Definition in maven.xml
        \item Start von der Konsole
        \item Start aus anderen Goals heraus
    \end{itemize}
    \begin{Beispiel}
        \begin{alltt}
<goal name="site-update" description="Updates the web site automatically">
    <cvs command="-q update -Pd"/>
    <attainGoal name="clean"/>
    <attainGoal name="site:deploy"/>
</goal>
        \end{alltt}
    \end{Beispiel}
}

\subsection[POM]{Project Object Model}
\frame{
    \frametitle{Project Object Model}
    \begin{itemize}
        \item Jedes Projekt hat sein eigenes POM
        \item Beschreibt das Projekt (Metainformationen)
        \item Projekt-Abh�ngigkeiten zu Libraries
        \item Definition des Build-Prozesses
    \end{itemize}
}

\subsection{Properties}
\frame{
    \frametitle{Konfigurierbarkeit}
    (Fast) alle Build- und Plugin-Parameter lassen sich anpassen.
    \begin{itemize}
        \item Definition in Property-Files
        \item{Vererbung der Properties
            \begin{enumerate}
                \item \begin{alltt}plugin.properties\end{alltt}
                \item \begin{alltt}\${maven.home}/bin/driver.properties\end{alltt}
                \item \begin{alltt}\${project.home}/project.properties\end{alltt}
                \item \begin{alltt}\${project.home}/build.properties\end{alltt}
                \item \begin{alltt}\${user.home}/build.properties\end{alltt}
            \end{enumerate}
        }
    \end{itemize}
    Lokale Anpassungen nur in den letzten beiden!
}

\subsection[xdoc]{Dokumentation}
\frame[all:1]{
    \frametitle{Dokumentation}
    S�mtliche Projektdokumentation in \bf{xdoc}
    \begin{itemize}
        \item Erweitertes (X)HTML
        \item muss well-formed sein
        \item kann jedoch nicht validiert werden
        \item erm�glicht einfache Gliederung
    \end{itemize}
    \begin{Beispiel}
        \scriptsize\begin{alltt}
<document>
  <properties>
    <title>Overview</title>
  </properties>
  <body>
    <section name="Overview">
      <p>The project goal is to develop a graphical user interface
         for programming a CPLD. The software is written in C++ and 
         based on QT.
      </p>
      ...
        \end{alltt}
    \end{Beispiel}
}

\section{Installation}
\frame{
    \frametitle{Installation}
maven genapp
Verzeichnisstruktur
}
\section{Konzepte}
% 3 Kerndateien
% TARGET=GOAL (post/preGoal)
%A project is described with
%a XML Project Object
%Model (POM)
% The POM defines how to
%build a project and the
%external dependencies
% The Maven functionality is
%implemented in terms of
%plugins
% The plugins are written in
%Jelly
% JARs are downloaded
%from a remote repository
%and stored into a local
%repository
\subsection{Architektur}
%grafik
\subsection[POM]{Project Object Model}
% projekt eigenschaften
\subsection[PP]{Project Properties}
% customizing von plugins
\subsection{Repositories}

\section{Beispiel}

%%% Local Variables: 
%%% TeX-master: "index"
%%% End: 
