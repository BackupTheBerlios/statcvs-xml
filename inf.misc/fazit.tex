\section[Tipps]{Tipps}
\subsection[Weitere Goals]{Weitere Hilfreiche Goals}

\frame[all:1]{
  \frametitle{Weitere Hilfreiche Goals}

  \begin{itemize}
  \item console: Konsoleneingabe zum Aufruf von Goals
    \begin{itemize}
    \item (+) schnelles sukessives Aufrufen von Goals
    \item (-) unzureichende Speicherfreigabe f�hrt zu OutOfMemory
      exception
    \end{itemize}
  \item eclipse: Generierung von Eclipse Projekt Dateien
    \begin{itemize}
    \item Ber�cksichtigung aller Dependencies
    \item Anlegen der MAVEN\_HOME Variable in Eclipse notwendig
    \end{itemize}
  \item jbuilder: Generierung von JBuilder Projekt Dateien
  \item plugin
  \item release
  \end{itemize}
}

\subsection{Command Line}

\frame[all:1]{
  \frametitle{Command Line Optionen}
  
  \begin{itemize}
  \item \begin{alltt}-g\end{alltt} Eine Liste aller Goals anzeigen
  \item -o Offline Modus. Nur Goals ausf�hren, die keine Internet
    Verbindung ben�tigen
  \item -p Project Datei angeben. N�tzlich f�r aufrufe aus nicht Projekt
    Verzeichnissen (z.B. in einem cron job).
  \end{itemize}
}

\subsection{Customizing}

\frame[all:1]{
  \frametitle{Customizing}
  
  \begin{itemize}
  \item Jelly: Code und Layout bunt durcheinander gemischt
  \item xnap.jsl Ein Beispiel f�r ein 3 Column Layout
  \item Includieren von php Skripten 
  \end{itemize}

\begin{alltt}
<jsl:template match="include" trim="false">
  <x:set var="_file" select="string(@file)"/>
  <jsl:comment>#include virtual="${_file}"</jsl:comment>
</jsl:template>
\end{alltt}

\begin{alltt}
<section name="FAQ">
<p><include file="faq.php"/></p>
\end{alltt}

}

\section{Zusammenfassung}

\frame{
  \frametitle{Zusammenfassung}

  \begin{itemize}
  \item Maven ist ein tolles Tool
    \begin{itemize}
    \item Sehr einfache Erstellung von Projekt Seiten
    \item Sehr einfache Erhebung von Projekt und Prozess Metriken
    \item Sehr einfaches Build und Release
    \item Sehr einfaches Deployment 
    \item Automatisches Aufl�sen von Dependencies
    \end{itemize}  
  \item Manches ist noch nicht ganz ausgereift
    \begin{itemize}
    \item Lange Startzeiten (auf marvin ca. 16s)
    \item Schwieriges Customizing der Webseiten wegen Jelly
    \item Keine unterst�tzung f�r I18n
    \item Viele Reports sind Java spezifisch
    \item Ohne Ende Leerzeilen in den generierten Webseiten
    \end{itemize}
  \end{itemize}  
}
  

%%% Local Variables: 
%%% TeX-master: "index"
%%% End: 
